\setcounter{section}{3}
\begin{tieude}
{23}{32-34}{Phương sai và độ lệch}{20/06/2022}
\end{tieude}
\phan{MỤC TIÊU.}
\subsubsection{Kiến thức.}
\begin{enumerate}
\item Hs hiểu được bảng phân bố tần số, tần suất ghép lớp.
\item Hs hiểu được ý nghĩa của các số đặc trưng của mẫu dữ liệu: số trung bình; phương sai; độ lệch chuẩn.
\end{enumerate}
\subsubsection{Kỹ năng.}
\begin{enumerate}
\item Hs biết lập bảng phân bố tần số, tần suất ghép lớp.
\item Hs biết tìm các số đặc trưng dựa vào bảng phân bố tần số, tần suất ghép lớp.
\end{enumerate}
\subsubsection{Thái độ:}
\begin{enumerate}
	\item Rèn luyện tư duy logic, thái độ nghiêm túc.
	\item Biết vận dụng kiến thức về thống kê vào cuộc sống.
	\item Diễn đạt các vấn đề toán học mạch lạc, phát triển tư duy và sáng tạo.
	\item Chủ động phát hiện, chiếm lĩnh tri thức mới, biết quy lạ về quen, có tinh thần hợp tác xây dựng cao.
\end{enumerate}
\subsection{Chuẩn bị của giáo viên và học sinh}
\subsubsection{Giáo viên:} 
\begin{enumerate}
	\item Giáo án, phiếu học tập, phấn, thước kẻ, máy chiếu,\ldots
	\item Kế hoạch bài học.
\end{enumerate}
\subsubsection{Học sinh:} 
\begin{enumerate}
	\item Sách giáo khoa, vở ghi, các đồ dùng học tập, \ldots
	\item Ôn tập lại kiến thức cũ, xem trước bài học hôm nay.
\end{enumerate}
\phan{Tiến trình dạy học}
\muccon{Khởi động}
\begin{kt}{Giới thiệu bảng phân bố tần số, tần suất ghép lớp.}
\textit{(1) Mục tiêu:} Lập được bảng phân bố tần số, tần suất ghép lớp khi cho trước số liệu thu thập hoặc bảng số liệu.\\
\textit{(2) Phương pháp/Kĩ thuật dạy học:} Vấn đáp, tự đọc sách.\\
\textit{(3) Hình thức tổ chức hoạt động:} Hoạt động theo cá nhân, hoạt động theo nhóm nhỏ.\\
\textit{(4) Phương tiện dạy học:} Có thể sử dụng Phiếu bài tập hoặc máy chiếu để chiếu nhanh câu hỏi.\\
\textit{(5) Sản phẩm:} Bảng phân bố tần số, tần suất ghép lớp.
\end{kt}
\begin{bang}{2}
\gvhs
{Gv giới thiệu bảng phân bố tần số, tần suất ghép lớp} 
{Hs theo dõi bài.}
\gvhs
{Gv: yêu cầu Hs làm  dựa vào  để lập bảng phân bố tần số, tần suất ghép lớp}
{Hs: làm bài theo sự hướng dẫn của Gv.}
\end{bang}
\muccon{Hình thành kiến thức}
\begin{kt}{CÁC SỐ ĐẶC TRƯNG.}
\textit{(1) Mục tiêu:} Hiểu và tính được các số đặc trưng của bảng số liệu.\\
\textit{(2) Phương pháp/Kĩ thuật dạy học:} Vấn đáp, tự đọc sách.\\
\textit{(3) Hình thức tổ chức hoạt động:} Hoạt động theo cá nhân, hoạt động theo nhóm nhỏ.\\
\textit{(4) Phương tiện dạy học:} Có thể sử dụng Phiếu bài tập hoặc máy chiếu để chiếu nhanh câu hỏi.\\
\textit{(5) Sản phẩm:} Các số đặc trưng.
\end{kt}

\begin{bang}{3}
\gvhsnd
{5'}
{Gv: giới thiệu cho hs công thức tính giá trị trung bình.\newline
-Hs: theo dõi, ghi bài.\newline }
{{\bf *Giá trị trung bình.}\newline
 $\overline{x}=\dfrac{1}{n}(n_1x_1+n_2x_2+\cdots +n_kx_k) \newline \phantom{x1}
 =f_1x_1+f_2x_2+\cdots +f_kx_k$ 
}
\gvhsnd{5'}{-Gv: giới thiệu cho hs công thức tính giá trị độ lệch chuẩn. \newline 
 -Gv: hướng dẫn cho hs tính \textbf{độ lệch chuẩn} $s_x=\sqrt{s^2_x}$\newline 
-Hs: theo dõi, ghi bài.\newline }
{{\bf *Công thức phương sai.}
\newline
 $s^2_x=\dfrac{1}{n}\left[n_1(x_1-\overline{x})^2+\cdots \right.$ \newline \hspace*{1.6 cm}  
$\left.  \cdots +n_k(x_k-\overline{x})^2\right]$  \newline
 $=f_1(x_1-\overline{x})^2+f_2(x_2-\overline{x})^2+\cdots \newline \hspace*{1.6 cm}  
\cdots +f_k(x_k-\overline{x})^2$}
\end{bang}